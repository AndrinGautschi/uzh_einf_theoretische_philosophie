\documentclass[../main.tex]{subfiles}
\begin{document}

\section{Generelle Informationen}
Diese Zusammenfassung umfasst Dr. Peter Schulte's Vorlesung <<Einführung in die theoretische Philosophie>> (160-003a), die ich im Herbstsemester 2022 besucht habe. Alle hier vorzufindenden Informationen sind in den Vorlesungen oder den zugehörigen Folien enthalten. Die Lektürekurstexte sind nicht Teil dieser Zusammenfassung. 

\section{Plagiate}
Ich habe alle Diagramme, Texte und andere Ressourcen selbst geschrieben (paraphrasiert) oder erstellt. Wo dies nicht der Fall ist (eine Hand voll Bilder), habe ich die Quelle verlinkt. 

\section{Gebrauch (Lizenz)}

 Der Gebrauch dieser Zusammenfassung ist erlaubt, solange damit kein Geld erwirtschaftet wird und ihr eins der folgenden Dinge tut:
\begin{itemize}
	\item ihr veröffentlicht eure Unterlagen zu einem Thema,
	\item helft mit, diese Vorlesung zu verbessern mittels Pull-Requests und Issues auf dem zugehörigen Githup-Repository, zu finden unter https://github.com/AndrinGautschi, oder
	\item gebt mir ein Bier aus (ja, die sind in Zürich für Studenten super teuer) mittels folgendem QR-Code:

	\begin{minipage}[t]{\linewidth}
          \raggedright
          \adjustbox{valign=t}{
            \includegraphics[width=.8\linewidth]{images/Twint_Sticker.png}
          }
    \end{minipage}
\end{itemize}

\section{Benutzte Tools}
Geschrieben wurde diese Zusammenfassung mittels Latex und dem genialen \href{https://www.texifier.com/}{Texifier App}. Mindmaps wurden in \href{https://simplemind.eu/}{SimpleMind Pro} erstellt.   

\section{Meine Philosophie}
In einer zunehmend gespalteneren Welt, in der die marktdiktierte Doktrin des egoistischen, nutzenmaximierenden Menschen gelebt und verbreitet wird und in der alles — selbst der Altruismus — zur Dienstleistung verkümmert, sind wir es, die dagegen halten! Wir tun dies, indem wir unsere Arbeiten frei zur Verfügung stellen, einander helfen und uns gegenseitig wertschätzen. Denn nur als Kollektiv können wir die Veränderung, hin zu einer lebenswerteren Welt, herbeiführen. 
\end{document}