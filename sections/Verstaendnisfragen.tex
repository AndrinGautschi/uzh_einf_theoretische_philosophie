\documentclass[../main.tex]{subfiles}
\begin{document}

\section{Personale Identität} % TODO fill in personale identität
\begin{enumerate}
	\item 
\end{enumerate}

\section{Sprache und Bedeutung}
\begin{enumerate}
	\item Was ist Sprachphilosophie? Nennen Sie einige ihrer Grundfragen.
	\item Wie unterscheidet sich der Gebrauch eines Ausdrucks von seiner Erwähnung?
	\item Was versteht man unter „nicht-natürlicher“ und „natürlicher“ Bedeutung (im Sinne von Grice)?
	\item Welche Arten von Worin unterscheiden sich Syntax, Semantik und Pragmatik?
	\item Was ist die Gegenstandstheorie der Bedeutung?
	\item Nennen sie einige Probleme, mit denen die Gegenstandstheorie konfrontiert ist
	\item Was ist die Gebrauchstheorie der Bedeutung?
\end{enumerate}

\section{Sprache und Realität} 
\begin{enumerate}
	\item Wie unterscheidet Frege zwischen ‚Sinn‘ und ‚Bedeutung‘?
	\item Welche Probleme der einfachen referentiellen Theorie der Bedeutung kann Frege durch seine Unterscheidung vermeiden?
	\item Zu Eigennamen:
	\begin{itemize}
		\item Was unterscheidet Deskriptivismus und Theorien des direkten Bezugs?
		\item Was sind Vorteile und Nachteile der beiden Ansätze?
	\end{itemize}
\end{enumerate}

\section{Wahrheit} 
\begin{enumerate}
	\item Welche Fragen beantwortet eine philosophische Wahrheitstheorie?
	\item Was sind verschiedene Konzeptionen von Wahrheitswertträgern?
	\item Welche anti-realistischen Wahrheitstheorien gibt es?
	\item Worin besteht der Unterschied zwischen substantiellen und deflationären Wahrheitstheorien?
	\item Was besagt der ‚alethische Realismus‘?
	\item Welchen Problemen sieht sich die Korrespondenztheorie gegenüber?
	\item Was ist die Kernthese von deflationären Wahrheitstheorien?
\end{enumerate}

\section{Universalien 1} 
\begin{enumerate}
	\item Was ist numerische bzw. qualitative Identität?
	\item Worin besteht das Problem der Universalien?
	\item Was sind Universalien?
	\item Wie lautet die Antwort des Universalienrealismus auf das Problem der Universalien?
	\item Was sind Tropen?
	\item Wie lautet die Antwort der Tropentheorie auf das Problem der Universalien?
\end{enumerate}

\section{Universalien 2} 
\begin{enumerate}
	\item Was besagen der Prädikatennominalismus, der Konzeptualismus, und der Klassennominalismus\textsubscript{N}?
	\item Welche Einwände kann man gegen diese Theorien erheben?
	\item Was besagt der radikale Nominalismus?
	\item Was unterscheidet den radikalen Nominalismus von den zuvor diskutierten klassischen nominalistischen Theorien?
	\item Was ist ein zentrales Problem für den radikalen Nominalismus?
\end{enumerate}

\section{Wissen und Skeptizismus 1} 
\begin{enumerate}
	\item Was ist die klassische Analyse des Wissens? (Was sind die notwendigen und hinreichenden Bedingungen für Wissen gemäß der klassischen Analyse?)
	\item Was ist propositionales Wissen?
	\item Warum ist die Rechtfertigung einer Überzeugung eine notwendige Bedingung für Wissen in der klassischen Analyse?
	\item Was ist Agrippas Trilemma?
	\item Wie funktioniert das Argument des nicht-feststellbaren skeptischen Szenarios?
	\item Was ist Fundamentalismus (in der Erkenntnistheorie)? Was unterscheidet Fundamentalismus von Kohärentismus?
\end{enumerate}

\section{Wissen und Skeptizismus 2} 
\begin{enumerate}
	\item Wie lautet die Mooreanische Antwort auf das Argument vom nicht-feststellbaren skeptischen Szenario?
	\item Was ist ein Beispiel für einen „Gettier-Fall“? Warum sind Gettier-Fälle ein Problem für die klassische Analyse des Wissens?
	\item Welche Arten des epistemischen Internalismus (in Bezug auf Rechtfertigung) gibt es? Was ist der Grundgedanke, der sie verbindet?
	\item Welche Arten des epistemischen Externalismus gibt es?
	\item Was ist ein Beispiel für eine externalistische Theorie der Rechtfertigung?
\end{enumerate}

\section{Wissen und Skeptizismus 3} 
\begin{enumerate}
	\item Was ist Reliabilismus in der Erkenntnistheorie?
	\item Wie können Reliabilist*innen auf skeptische Argumente (Agrippas Trilemma und das Argument des nicht-feststellbaren skeptischen Szenarios) reagieren?
	\item Was ist Nozicks Sensitivitätstheorie des Wissens?
	\item Wie lautet Nozicks Antwort auf das Gettier-Problem?
	\item Was ist das Problem des Wertes des Wissens (allgemein formuliert)?
	\item Warum scheint der Reliabilismus das Problem des Wertes des Wissens nicht lösen zu können?
\end{enumerate}

\section{Dualismus} 
\begin{enumerate}
	\item Was ist das Körper-Geist-Problem?
	\item Was besagt der Substanzdualismus?
	\item Worin unterscheidet sich die dualistische Position, die Descartes vertritt, von der dualistischen Position Platons?
	\item Welche Argumente führt Descartes für den Substanzdualismus ins Feld? Was sind die Schwachpunkte dieser Argumente?
	\item Was versteht man unter dem ‚Interaktionsproblem‘ für den Substanzdualismus? Was sind die verschiedenen Aspekte dieses Problems?
	\item Welche anderen Formen des Dualismus gibt es neben dem interaktionistischen Substanzdualismus?
\end{enumerate}

\section{Materialismus} 
\begin{enumerate}
	\item Was besagt die Identitätstheorie?
	\item Was ist Smarts zentrales Argument für die Identitätstheorie? Wie begründet er die Prämissen des Arguments?
	\item Wie argumentiert Lewis für die Identität von mentalen und physischen Zuständen?
	\item Warum wird Lewis’ Position heute als eine Version des „Funktionalismus“ bezeichnet?
\end{enumerate}

\section{Können Computer denken?} 
\begin{enumerate}
	\item Was ist ein Computer?
	\item Wie funktioniert eine Turing-Maschine?
	\item Was ist der Grundgedanke von computationalen Theorien des Geistes? Was unterscheidet starke von schwachen Versionen dieser Theorie?
	\item Wie funktioniert das Gedankenexperiment vom chinesischen Zimmer?
	\item Wie argumentiert Searle gegen die Möglichkeit von denkenden digitalen Computern?
\end{enumerate}

\section{Was ist Kunst?} 
\begin{enumerate}
	\item Welche Varianten der Darstellungstheorien der Kunst gibt es?
	\item Was ist der Kerngedanke einer Ausdruckstheorie der Kunst?
	\item Was ist der Unterschied zwischen Essentialismus und Anti- Essentialismus in der Ästhetik?
\end{enumerate}



\end{document}