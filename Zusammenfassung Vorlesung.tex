\title{Einf\"uhrung Theoretische Philosophie\\\underline{Zusammenfassung}}
\author{
	Andrin Gautschi \\
	Universit\"at Z\"urich, \underline{Schweiz}\\
}
\date{\today}

\documentclass[12pt]{report}

% Fancier enumerations
\usepackage{enumitem}[listparindent=0.7cm]

% Some math functionality
\usepackage{mathtools}

% Hyperlinks
\usepackage{hyperref}

% Graphics
\usepackage{graphicx}
\graphicspath{{./images/}}
\usepackage{adjustbox}

% Correct positioning of tables
\usepackage{flafter} 

% Splitting and referencing subfiles
\usepackage{subfiles}

% Colored boxes inclusive some default boxes
\usepackage{tcolorbox}
\newtcolorbox{examplebox}{colback=blue!5!white, colframe=blue!75!black}
\newtcolorbox{infobox}{colback=blue!5!white, colframe=blue!75!black}
\newtcolorbox{warningbox}{colback=orange!5!white, colframe=orange!75!black} \newtcolorbox{errorbox}{colback=red!5!white, colframe=red!75!black}

% Encoding, language and time settings
\usepackage[utf8]{inputenc}
\usepackage[ngerman]{babel}
\usepackage{datetime}


% Page settings
\usepackage[paper=a4paper,top=1in,bottom=1in,right=0.5in,left=1.5in]{geometry}
\usepackage{lipsum}
\usepackage{txfonts}
\usepackage[T1]{fontenc}
\usepackage{titlesec}
\titleformat{\chapter}
  {\Large\bfseries} % format
  {}                % label
  {0pt}             % sep
  {\huge}           % before-code


\begin{document}

\newgeometry{margin=1in}
\begin{titlepage}
	\begin{center}
        \vspace*{9cm}
        \Huge\textbf{Zusammenfassung}\\
        \vspace{0.5cm}
        \LARGE Einführung Theoretische Philosophie   
        \vspace{2cm}\\
        \textbf{Andrin Gautschi}
        \vfill
        \Large
        Philosophie\\
        Universität Zürich\\
        Schweiz\\
        \today
    \end{center}
\end{titlepage}

\newgeometry{top=1.2in, left=0.9in}

\tableofcontents
\newpage


\chapter{Benutzungsanleitung}
\subfile{sections/Benutzungsanleitung.tex}

\chapter{Einführung}
\subfile{sections/Einfuehrung.tex}

\chapter{Personale Identität}
\subfile{sections/Personale_Identitaet.tex}

\chapter{Sprache und Bedeutung}\label{ChapterSpracheUndBedeutung}
\subfile{sections/Sprache_und_Bedeutung.tex}

\chapter{Sprache und Realität}
\subfile{sections/Sprache_und_Realitaet.tex}

\chapter{Wahrheit}
\subfile{sections/Wahrheit.tex}

\chapter{Universalien}
\subfile{sections/Universalien.tex}

\chapter{Wissen und Skeptizismus}
\subfile{sections/Wissen_und_Skeptizismus.tex}

\chapter{Dualismus}
\subfile{sections/Dualismus.tex}

\chapter{Materialismus}
\subfile{sections/Materialismus.tex}

\chapter{Können Computer denken?}
\subfile{sections/Koennen_Computer_denken.tex}

\chapter{Was ist Kunst?}
\subfile{sections/Was_ist_Kunst.tex}

\chapter{Verständnisfragen}
\subfile{sections/Verstaendnisfragen.tex}
\end{document}